% Created 2018-08-07 Di. 15:12
\documentclass[11pt]{article}
\usepackage[utf8]{inputenc}
\usepackage[T1]{fontenc}
\usepackage{fixltx2e}
\usepackage{graphicx}
\usepackage{longtable}
\usepackage{float}
\usepackage{wrapfig}
\usepackage{rotating}
\usepackage[normalem]{ulem}
\usepackage{amsmath}
\usepackage{textcomp}
\usepackage{marvosym}
\usepackage{wasysym}
\usepackage{amssymb}
\usepackage{hyperref}
\tolerance=1000
\usepackage{ngerman}
\author{Richard Stewing}
\date{07.08.2018}
\title{Beispiel Kleisli Kategorie}
\hypersetup{
  pdfkeywords={},
  pdfsubject={},
  pdfcreator={Emacs 25.3.1 (Org mode 8.2.10)}}
\begin{document}

\maketitle

\section{Definition Kleisli}
\label{sec-1}

Sei $\mathcal{C}$ eine Kategorie und $M=(T, \mu, \eta)$ eine Monade. 
Mit $T: \mathcal{C} \to \mathcal{C}$ als Endofunktor und 
$\mu: T(T(A)) \to T(A)$ und $\eta: A \to T(A)$ die \href{https://de.wikipedia.org/wiki/Monoid}{Monoid}-Operationen.
Dann ist die Kleisli-Kategorie $\mathcal{C}_M$ definiert mit:
\begin{itemize}
\item $Obj(\mathcal{C}_M) = Obj(\mathcal{C}_M)$
\item $Mor_{\mathcal{C}_M}(X,Y) = Mor_{\mathcal{C}}(X, T(Y))$
\end{itemize}
und 
\begin{itemize}
\item $id_A = \eta_A$
\item $f \circ_{\mathcal{C}_M} g = \mu \circ_{\mathcal{C}} T(f) \circ_{\mathcal{C}} g$
\end{itemize}

\section{Beispiel mit der Listen-Monad $\_^*$}
\label{sec-2}
\begin{itemize}
\item Sei $A$ eine Menge, $A^*$ ist die Menge aller W"orter "uber $A$
\item Sei $f:A \to B$, dann ist $f^*([a_0,\dots]) = [f(a_0),\dots] \in B^*$ (wie das Haskell \emph{map})
\item $\eta:A \to A^*, \eta(a) = [a]$
\item $\mu: A^{**} \to A^*, \mu = concat$
\end{itemize}
% Emacs 25.3.1 (Org mode 8.2.10)
\end{document}
